\chapter{Вступление}
\label{cha:intro}
Картина является общей формой искусства и средством коммуникации. Нередко для их рисования художники применяют повторяющиеся операций, такие как мазки, штрихи или пунктиры. Такие приемы являются неотъемлемой частью художественного стиля и композиции. Повторение является неотъемлемой частью природы, проявляющейся в общих явлениях, таких как поверхностные узоры (например, стены, ткани, полы), геометрические структуры (например, галька, ветви) и деятельность человека (например, рисование, жестикуляция, моделирование). Тем не менее повторяющиеся действия утомительны для ручного труда и довольно универсальны, что позволяет генерировать их автоматически. 
Повторение было важным предметом изучения инженерных и научных дисциплин из-за повсеместности. Создаваемый алгоритм представляет собой интерактивную систему для анализа и синтеза повторений эскизов художника. Будет создана интерактивная система цифровой живописи для автодополнения повторений, таких как штриховка и пунктир, сохраняющая при этом новаторские вариации и сохраняющая естественные рабочие процессы. В отличие от предыдущих работ, посвященных статическим и конечным штрихам, система анализирует рабочий процесс и обеспечивает пользователя высококачественными подсказками, ориентированными на контекст. Пользователи смогут рисовать в обычной манере, в то же время система будет автоматически предоставлять и обновлять предложения интерактивно без каких-либо дополнительных действий. Пользователи смогут игнорировать или принимать эти предложения, аналогично функциям автодополнения в интегрированных средах разработки программирования, тем самым сохраняя полный контроль над процессом рисования. В последствии алгоритм будет улучшен и сможет обрабатывать структуры высокого уровня.