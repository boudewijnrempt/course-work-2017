\chapter{Вступление}
\label{cha:intro}
Повторение является неотъемлемой частью природы, проявляющейся в общих явлениях, таких как поверхностные узоры (например, стены, ткани, полы), геометрические структуры (например, галька, ветви), динамические движения (например, турбулентность жидкости, ходовые циклы, движение толпы) и деятельность человека (Например, рисование, жестикуляция, моделирование). Повторение было важным предметом изучения многих инженерных и научных дисциплин из-за его повсеместности. Главная задача заключается в разработке общих и эффективных методов и простых в использовании интерфейсах для различных явлений и областей приложений. Этот алгоритм представляет собой интерактивную системы для анализа и синтеза повторений эскизов художника.
Я представляю интерактивную систему цифровой живописи для автодополнения утомительных повторений, таких как штриховки и пунктиры, сохраняя при этом новаторские вариации и сохраняя естественные потоки. В отличие от предыдущих работ, посвященных статическим и конечным штрихам, моя система анализирует рабочий процесс динамического и промежуточного рисования, который позволяет моей системе понять, как штрихи рисуются в прошлом, чтобы обеспечить высококачественные подсказки, ориентированные на контекст. Пользователи могут рисовать с моей системой как обычно, в то время как моя система автоматически предоставляет и обновляет предложения интерактивно без каких-либо дополнительных действий. Пользователи могут игнорировать или принимать эти предложения, аналогичные функциям автозаполнения в интегрированных средах разработки программирования, тем самым сохраняя полный контроль над процессом рисования.