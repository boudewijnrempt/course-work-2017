\chapter{Аналитический раздел}
\label{cha:analysis}
%
% % В начале раздела  можно напомнить его цель
%
The quick brown fox jumps over the lazy dog.  The quick brown fox jumps over the lazy dog.  The quick brown fox jumps over the lazy dog.  The quick brown fox jumps over the lazy dog.  The quick brown fox jumps over the lazy dog.The quick brown fox jumps over the lazy dog.  The quick brown fox jumps over the lazy dog.  The quick brown fox jumps over the lazy dog.  The quick brown fox jumps over the lazy dog.  The quick brown fox jumps over the lazy dog.The quick brown fox jumps over the lazy dog.  The quick brown fox jumps over the lazy dog.  The quick brown fox jumps over the lazy dog.  The quick brown fox jumps over the lazy dog.  The quick brown fox jumps over the lazy dog.The quick brown fox jumps over the lazy dog.  The quick brown fox jumps over the lazy dog.  The quick brown fox jumps over the lazy dog.  The quick brown fox jumps over the lazy dog.  The quick brown fox jumps over the lazy dog.The quick brown fox jumps over the lazy dog.  The quick brown fox jumps over the lazy dog.  The quick brown fox jumps over the lazy dog.  The quick brown fox jumps over the lazy dog.  The quick brown fox jumps over the lazy dog.The quick brown fox jumps over the lazy dog.  The quick brown fox jumps over the lazy dog.  The quick brown fox jumps over the lazy dog.  The quick brown fox jumps over the lazy dog.  The quick brown fox jumps over the lazy dog.The quick brown fox jumps over the lazy dog.  The quick brown fox jumps over the lazy dog.  The quick brown fox jumps over the lazy dog.  The quick brown fox jumps over the lazy dog.  The quick brown fox jumps over the lazy dog.


% Обратите внимание, что включается не ../dia/..., а inc/dia/...
% В Makefile есть соответствующее правило для inc/dia/*.pdf, которое
% берет исходные файлы из ../dia в этом случае.


%%% Local Variables:
%%% mode: latex
%%% TeX-master: "rpz"
%%% End:
