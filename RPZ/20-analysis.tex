\chapter{Аналитический раздел}
\label{cha:analysis}
%
% % В начале раздела  можно напомнить его цель
%
\section{Общие требования к алгоритму}
%\subsection{Suggestive Sketching}
Важный нюанс работы системы — это сохранение выразительности и достаточного контроля над процессом рисования для пользователя\cite{bib1}. Современные системы, которые работают на основании введенных данных, преуспели в этом(например \cite{bib2},\cite{bib3}), но в большинстве случаев они работают последовательно: сначала они получают на вход небольшие экземпляры, а затем клонируют их в желаемых областях вывода с помощью различного управления жестами. Такой последовательный режим может нарушить непрерывный и спонтанный характер живописи, так как пользователи не могут знать априори желаемые повторения и хотели бы поэкспериментировать в процессе рисования. Кроме того, такой больше подходит для больших и однородных по узору областей и для внесения разнообразия в рисунок может потребоваться чрезмерное большое количество действий. Таким образом, разрабатываемая система работает более естественно и не нарушает обычный процесс рисования. Она постепенно записывает нарисованные пользователем данные и анализирует их структуру и цветовые соотношения. Когда достаточное количество повторений обнаружено, система может спрогнозировать и автодополнить действия пользователя в ближайшее время, вокруг текущей области рисования, или через области, структурированные похожим образом. Кроме того, пользователь может принять, проигнорировать или изменить предсказания и таким образом полностью руководить процессом автодополнения.
\section{Автодополение повторов в живописи}
Ключевая задача алгоритма состоит в том, чтобы учитывать не только то, что было нарисовано, но и то, как оно нарисовано. Проводится контекстный анализ (\cite{bib23},\cite{bib24})взаимосвязей формы и цвета среди последних мазков и при помощи этого получается потенциальная информация для прогнозирования будущих мазков. В отличии от предыдущих систем рисования(например \cite{bib26},\cite{bib27}), которые требуют четкого порядка действий пользователя, разрабатываемый метод автоматически обнаруживает и поддерживает потенциальные связи между мазками в фоновом режиме. 
\section{Предыдущие работы }
\subsection{Создание дополняемых эскизов}
Рисование является обычной, и все же сложной человеческой деятельностью. Значительные усилия исследователей были направлены на проектирование наводящих или управляемых систем рисования, которые используют различные формы данных, чтобы помочь пользователям. Например, созданию набросков портрета может способствовать материал, собранный при помощи краудсорсинга(\cite{bib28}) или анализируемых данных лица(\cite{bib29}). Чтобы помочь пользователям создать большой набор предметов, Ли(\cite{bib17}) интерактивно дает пользователю подсказки, отображая тени, извлеченные из веб-изображений. В то же время Iarussi (\cite{bib30}) обеспечивает структурные наставления на основе художественных и геометрических принципов. Кроме статических изображений или рисунков, можно также учить новичков при помощи записанных процессов рисования опытных художников(\cite{bib18}).
Разрабатываемый метод следует этой концепции, но использует данные одного пользователя, чтобы помогать ему же в будущих рисунках.
\subsection{Эскизы на основе данных}
Хотя управляемые системы помогают пользователям рисовать проще, им по-прежнему приходится иметь дело с ручными повторениями. Для того, чтобы облегчить этот труд, многие системы были разработаны для автоматизации повторяющихся мазков с помощью вычислений основанных на данных. Ярким примером является детализированные текстуры или узоры (\cite{bib3}, \cite{bib31}, \cite{bib32}) создание которых может хорошо сочетается с традиционными стилями взаимодействия - копирования и вставки. Kazi (\cite{bib2}, \cite{bib11}) предлагает понятное управление жестами, чтобы помочь в создании как статических, так и динамических элементов.
Разрабатываемый метод также использует данные, для создания посказок, но использует динамические рабочие процессы, а не статические шаблоны или анимированные спрайты. 

\subsection{Quick, Draw!}
Quick, Draw! это онлайн-игра, разработанная компанией Google, которая бросает вызов игрокам: необходимо нарисовать изображение объекта или понятия, а затем система использует искусственный интеллект нейронной сети, чтобы угадать, что представляет собой рисунок. Искусственный интеллект, встроенный в игру учится на каждом рисунке, совершенствуя свою способность правильно угадывать. Понятия, которые он угадывает, могут быть простыми, например, «ноги» или более сложными, такими как «миграция животных».

По состоянию на июль 2017 года более 15 миллионов игроков внесли миллионы рисунков в память игры Quick, Draw! Эти эскизы являются уникальным набором данных, который может помочь разработчикам обучать новые нейронные сети, помогать исследователям видеть шаблоны в том, как люди во всем мире рисуют.

Разрабатываемый метод также использует данные пользователя для создания посказок, но на данном этапе использование нейронной сети излишне, так как систему можно реализовать более простыми методами.

\section{Алгоритм}
Формирования пространственных повторений локально похожи, но могут быть и весьма различны, это зависит от окружающего контекста, таких как окна с разными точками зрения. Для того, чтобы синтезировать предложения высокого качества, система должна быть способна автоматически определять чертежные повторы и анализировать контекстные элементы управления.
Основная идея системы заключается в том, чтобы обрабатывать повторяющиеся операции рисования как форму текстуры рабочего процесса, а также расширить типовые методы для представления, анализа и синтеза. Необходимо определять каждую операцию рисования в виде непрерывного движения пера не поднимая его. 

%\subsection{Обзор алгоритма}

%\subsection{Текстура рабочего процесса}

%\subsubsection{Образец}
%\subsubsection{Operation}
%\subsubsection{Neighborhood}

%\subsection{Workflow Synthesis}
%\subsubsection{Initialization}
%\subsubsection{SearchStep}
%\subsubsection{Assignment Step}
%\subsubsection{Weighting}

%\subsection{Workflow Analysis}
%\subsubsection{Local Analysis}
%\subsubsection{Global Analysis}
%\subsubsection{Prediction Quality}
%\subsubsection{Prediction}

%\subsection{Implementation}

%\section{User Evaluation}
%\subsection{Study Protocol}
%\subsection{Target Sketch Performance}
%\subsection{Open Sketch Results}


% Обратите внимание, что включается не ../dia/..., а inc/dia/...
% В Makefile есть соответствующее правило для inc/dia/*.pdf, которое
% берет исходные файлы из ../dia в этом случае.
%$  $

%%% Local Variables:
%%% mode: latex
%%% TeX-master: "rpz"
%%% End:
