\chapter{Аналитический раздел}
\label{cha:analysis}
%
% % В начале раздела  можно напомнить его цель
%
\section{Previous Work}
\subsection{Suggestive Sketching}
Drawing is a common and yet challenging human activity. Significant research efforts have been devoted to design suggestive or guided drawing systems that use various forms of data to help users. For example, portrait sketching can be assisted by crowd- sourced sketches \cite{bib1} or analyzed face data \cite{bib1}. To help users draw a larger collection of objects, Lee et al. \cite{bib1} interactively guides users’ progress by displaying shadows extracted from web images, while Iarussi et al. \cite{bib1} provide structural guidances based on artistic and geometric principles. Besides static images or drawings, it is also possible to guide novices through recorded workflows of experienced painters \cite{bib1}.
Our method follows this line of work, but uses past drawing workflows from the same user to guide future drawings.

\subsection{Data-driven Sketching}
Although the guided systems can help users draw easier, they still have to deal with manual repetitions. To help ameliorate such tedious process, many systems have been designed to automate the authoring of repetitive drawings via data driven computation. One prime example is the detailed textures or patterns  \cite{bib1} \cite{bib1} \cite{bib1} \cite{bib1} whose creation can fit well with the traditional copy-and-paste interaction styles. Kazi et al.  \cite{bib1} \cite{bib1} provide friendly gesture controls to help create both static and dynamic elements.
Our method also follows a data-driven approach to help creating repetitions, but uses dynamic workflows rather than static patterns or animated sprites.

\subsection{Using Workflows}
Chimera \cite{bib1} is an old system for recording and editing graphical histories, specifically the repetitive operations. Recent years have seen the rise of methods that utilize workflows in various forms, such as exploration \cite{bib1}, visualization  \cite{bib1} \cite{bib1} \cite{bib1} \cite{bib1}, stylization \cite{bib1}, beautification \cite{bib1}, tutorial  \cite{bib1} \cite{bib1}, and revision \cite{bib1}. These methods often rely on pre-recorded workflows, which, when not available, may also be recovered through a certain extent of analysis  \cite{bib1} \cite{bib1} \cite{bib1} \cite{bib1}. Recently, Nancel et al. \cite{bib1} provides a comprehensive survey of different conceptual models for workflow analysis.
We follow this line of work, but focus on the analysis and synthesis of dynamic workflows for suggestive painting repetitions.

\section{User Interface}
\subsection{Future Prediction}
\subsection{Edit Propagation}
\subsection{Workflow Clone}

\section{Method}

\subsection{Method Overview}

\subsection{Workflow Texture}
\subsubsection{Sample}
\subsubsection{Operation}
\subsubsection{Neighborhood}

\subsection{Workflow Synthesis}
\subsubsection{Initialization}
\subsubsection{SearchStep}
\subsubsection{Assignment Step}
\subsubsection{Weighting}

\subsection{Workflow Analysis}
\subsubsection{Local Analysis}
\subsubsection{Global Analysis}
\subsubsection{Prediction Quality}
\subsubsection{Prediction}

\subsection{Implementation}

\section{User Evaluation}
\subsection{Study Protocol}
\subsection{Target Sketch Performance}
\subsection{Open Sketch Results}

\section{Limitations and Future Works}

% Обратите внимание, что включается не ../dia/..., а inc/dia/...
% В Makefile есть соответствующее правило для inc/dia/*.pdf, которое
% берет исходные файлы из ../dia в этом случае.
%$  $

%%% Local Variables:
%%% mode: latex
%%% TeX-master: "rpz"
%%% End:
